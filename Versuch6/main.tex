\documentclass[10pt]{scrreprt}
\usepackage[utf8]{inputenc}
\usepackage{amsfonts}
\usepackage{amsmath}
\usepackage{amssymb}
\usepackage{commath}
\usepackage[ngerman]{babel}
\usepackage{enumitem}
\usepackage{booktabs}
\usepackage{longtable}
\usepackage{relsize}
\usepackage{pgfplots}
\usepackage{csvsimple}
\usepackage{pgfplotstable}
\usepackage{siunitx}
\usepackage{fancyhdr}
\usepackage{color}
\usepackage{float}
\usepackage{listings}

\definecolor{mygreen}{RGB}{28,172,0} % color values Red, Green, Blue
\definecolor{mylilas}{RGB}{170,55,241}


\lstset{language=Matlab,%
%basicstyle=\color{red},
breaklines=true,%
morekeywords={matlab2tikz},
keywordstyle=\color{blue},%
morekeywords=[2]{1}, keywordstyle=[2]{\color{black}},
identifierstyle=\color{black},%
stringstyle=\color{mylilas},
commentstyle=\color{mygreen},%
showstringspaces=false,%without this there will be a symbol in the places where there is a space
%numbers=left,%
%numberstyle={\tiny \color{black}},% size of the numbers
%numbersep=9pt, % this defines how far the numbers are from the text
emph=[1]{for,end,break},emphstyle=[1]\color{red}, %some words to emphasise
%emph=[2]{word1,word2}, emphstyle=[2]{style},
}

\setlength\parindent{0pt}

\setcounter{chapter}{5}
\setcounter{secnumdepth}{3}
\setcounter{figure}{12}


\pagestyle{fancy}
\fancyhf{}
\lhead{GPET Versuch 6}
\rhead{Tim Luchterhand, Paul Nykiel}
\cfoot{\thepage}

\author{Tim Luchterhand, Paul Nykiel \protect\\ tim.luchterhand@uni-ulm.de, paul.nykiel@uni-ulm.de}
\title{GPET Versuch 6 --- Karl Willy Wagner Versuch}
\subtitle{Gruppe: Dienstag14}

\begin{document}
    \maketitle
    \section{Bestimmung der Grenzfrequenz}
    \paragraph{Aufgabe}
    Bestimmen Sie die Grenzfrequenz eines RC-Tiefpasses ($R = 1\si{k\ohm}$ und $C = 2.2\si{\mu\farad}$) mit
    Hilfe des Oszilloskops. Berechnen Sie damit und mit der Formel aus Gl. 13 den Innenwiderstand des Oszilloskops.
    \begin{enumerate}
        \item Messen Sie mit dem Multimeter die exakten Werte $R$ und $C$ der Bauteile.
        \item Bauen Sie die Schaltung auf und messen Sie mit Hilfe des Oszilloskops die Grenzfrequenz
                $f_g$. Gehen Sie möglichst sorgfältig vor.
        \item Geben Sie eine geeignete Formel zur Bestimmung des Innenwiderstands des Oszilloskops an.
        \item Bestimmen Sie den Innenwiderstand für die gemessene Frequenz bei einem Messfehler
                (der Grenzfrequenz) von $\pm1\%$. Also $R_{i,f_g}$, $R_{i,f_g+1\%}$ und $R_{i,f_g-1\%}$.
        \item Bestimmen Sie den Innenwiderstand $R_i$ des Oszilloskops mit Hilfe des Multimeters.
        \item Wie groß ist der Innenwiderstand $R_i$ laut Beschreibung des Oszilloskops?
        \item Geben Sie ein kurzes Fazit für diesen Teil des Versuchs.
    \end{enumerate}

    \paragraph{Protokoll}
    %TODO Vorbereitung: Innenwiderstand Oszi, Formel
    \begin{enumerate}
        \item Mit Multimeter gemessenen Werte:
            \begin{eqnarray*}
                R &=& \si{\ohm}\\
                C &=& \si{\farad}
            \end{eqnarray*}
        \item
            \begin{equation*}
                f_g = \si{\hertz}
            \end{equation*}
        \item
            Der RC-Tiefpass ist durch den Innenwiderstand des Oszilloskops
            belastet, das heißt: $R_L = R_{innen}$. Daraus folgt:
            \begin{eqnarray*}
                A &=& \frac{1}{\sqrt{{\left(1+\frac{R}{R_L}\right)}^2 + {(\omega R C)}^2}}\\
                \Leftrightarrow
                R_i &=& R_L = \frac{R}{\sqrt{\dfrac{1}{A^2} -{(\omega R C)}^2} - 1}
            \end{eqnarray*}
        \item
            \begin{eqnarray*}
                R_{i,f_g} &=& \si{\ohm}\\
                R_{i,f_g+1\%} &=& \si{\ohm}\\
                R_{i,f_g-1\%} &=& \si{\ohm}\\
            \end{eqnarray*}
        \item
            Mit Multimeter gemessener Innenwiderstand:
            \begin{equation*}
                R_i = \si{\ohm}
            \end{equation*}
        \item
            Innenwiderstand laut Beschreibung:
            \begin{equation*}
                R_i = 1 \si{M\ohm} \pm 2\%
            \end{equation*}
        \item \textbf{Fazit:}
    \end{enumerate}

    \section{Zeitverhalten eines Hochpasses}
    \paragraph{Aufgabe}
    \begin{center}
        \begin{figure}[H]
            \includegraphics[width=\textwidth]{abb8.png}
            \caption{Hochpass erster Ordnung}
            \label{fig:abb8}
        \end{figure}
    \end{center}
    Messen Sie die Abfallzeit der Schaltung aus~\ref{fig:abb8} für folgende Bauteil- und
    Spannungswerte:
    \begin{eqnarray*}
        R &=& 100\si{\ohm}\\
        C &=& 2.2\si{\mu \farad}\\
            \underline{U_1} &=&
            \begin{cases}
                t < 0& -2.5\si{\volt}\\
                t \geq 0& 2.5\si{\volt}
            \end{cases}
    \end{eqnarray*}
    \begin{enumerate}
        \item Bauen Sie das Filter aus~\ref{fig:abb8} auf und messen Sie die Abfallzeit einmal von Hand
            und einmal mit Hilfe der eingebauten Rise-/Falltime Messung.
        \item Berechnen Sie den theoretisch zu erwartenden Wert.
    \end{enumerate}

    \paragraph{Protokoll}
    %TODO Formel

    \section{Bandpass}
    \paragraph{Aufgabe}
    \begin{center}
        \begin{figure}[H]
            \includegraphics[width=\textwidth]{abb17.png}
            \caption{Unbelasteter Bandpass erster Ordnung}
            \label{fig:abb17}
        \end{figure}
    \end{center}
    Im Folgenden soll das Übertragungsverhalten eines Bandpasses 1. Ordnung untersucht
    werden. Bauen Sie dazu das Bandpass-Filter aus~\ref{fig:abb17} mit $R = 100 \si{\ohm}$ und $C = 2.2 \si{\mu\farad}$
    auf. Als Induktivität verwenden Sie eine Spule mit 250 Windungen.
    Nehmen Sie die Übertragungsfunktion mit Hilfe der Matlab-GUI (100 Punkte, $1\si{\volt}_{pp}$,
    $100\si{\hertz}$–$100\si{k\hertz}$, log) auf und fügen Sie das Diagramm in Ihr Protokoll ein. Diskutieren
    Sie Ihre Ergebnisse.

    \paragraph{Protokoll}
    \begin{center}
        \begin{figure}[H]
            %\includegraphics[width=\textwidth]{abb19.png}
            \caption{Übertragungsfunktion des Bandpass}
            %\label{fig:abb19}
        \end{figure}
    \end{center}
    %TODO Ergebniss diskutieren

    \section{Bandsperre}
    \paragraph{Aufgabe}
    \begin{center}
        \begin{figure}[H]
            \includegraphics[width=\textwidth]{abb19.png}
            \caption{Unbelastete Bandsperre erster Ordnung}
            \label{fig:abb19}
        \end{figure}
    \end{center}
    Weiter soll das Übertragungsverhalten einer Bandsperre 1.Ordnung untersucht werden.
    Bauen Sie dazu die Bandsperre aus~\ref{fig:abb19} mit $R = 100 \si{\ohm}$ und $C = 2.2 \si{\mu\farad}$ auf. Als
    Induktivität verwenden Sie eine Spule mit 250 Windungen.
    Nehmen Sie die Übertragungsfunktion mit Hilfe der Matlab-GUI (100 Punkte, $R_{i,f_g+1\%}$,
    $100\si{\hertz}$–$100\si{k\hertz}$, log) auf und fügen Sie das Diagramm in Ihr Protokoll ein. Diskutieren
    Sie Ihre Ergebnisse.
    \paragraph{Protokoll}
    \begin{center}
        \begin{figure}[H]
            %\includegraphics[width=\textwidth]{abb19.png}
            \caption{Übertragungsfunktion der Bandsperre}
            %\label{fig:abb19}
        \end{figure}
    \end{center}
    %TODO Ergebniss diskutieren

    \section{Frequenzbereich und Audio-Filterung}
    \paragraph{Aufgabe}
    \begin{enumerate}
        \item Hören Sie sich zunächst das Audio-Sprach-Signal an. In welchem Frequenzbereich
            befindet sich die Störung?
        \item Realisieren Sie mit zwei Kondensatoren ($C = 2.2 \si{\mu\farad}$) und einer Spule (500 Windungen)
            ein Hochpass-Filter in T-Schaltung und zeigen Sie den Aufbau Ihrem Tutor.
        \item Bestimmen Sie den ohmschen Widerstand des Kopfhörers.
        \item Messen Sie die Übertragungsfunktion des Hochpass-Filter einmal ohne und einmal
            mit angeschlossener Last (Kopfhörer) und stellen Sie diese in zwei Diagrammen
            dar (200 Punkte, $100\si{\hertz}$–$100\si{k\hertz}$, log.). Verwenden Sie dazu die Matlab-GUI
            ($500\si{m\volt}_{pp}$). Speichern Sie die Plots und fügen Sie diese in Ihr Protokoll ein. Erklären
            Sie die Verläufe der Funktionen für kleine und große Frequenzen.
        \item Hören Sie sich einmal das ungefilterte und einmal das gefilterte Audio-Sprach-Signal
            an. Welche Person spricht da und welchen Satz sagt sie am Ende?
    \end{enumerate}

    \paragraph{Protokoll}
    \begin{enumerate}
        \item Die Störung befindet sich im Frequenzbereich von
        \item
            Innenwiderstand des Kopfhörers:
            \begin{equation*}
                R_i = \si{\ohm}
            \end{equation*}
        \item
            Erklärung
            \begin{center}
                \begin{figure}[H]
                    %\includegraphics[width=\textwidth]{abb19.png}
                    \caption{Übertragungsfunktion des Hochpass mit Last}
                    %\label{fig:abb19}
                \end{figure}
            \end{center}
            \begin{center}
                \begin{figure}[H]
                    %\includegraphics[width=\textwidth]{abb19.png}
                    \caption{Übertragungsfunktion des Hochpass ohne Last}
                    %label{fig:abb19}
                \end{figure}
            \end{center}
        \item Es spricht
    \end{enumerate}
\end{document}
