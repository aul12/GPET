\documentclass[10pt]{scrreprt}
\usepackage[utf8]{inputenc}
\usepackage{amsfonts}
\usepackage{amsmath}
\usepackage{amssymb}
\usepackage{commath}
\usepackage[ngerman]{babel}
\usepackage{enumitem}
\usepackage{booktabs}
\usepackage{longtable}
\usepackage{relsize}
\usepackage{pgfplots}
\usepackage{csvsimple}
\usepackage{pgfplotstable}
\usepackage{siunitx}
\usepackage{fancyhdr}
\usepackage{color}
\usepackage{float}
\usepackage{listings}

\definecolor{mygreen}{RGB}{28,172,0} % color values Red, Green, Blue
\definecolor{mylilas}{RGB}{170,55,241}


\lstset{language=Matlab,%
    %basicstyle=\color{red},
    breaklines=true,%
    morekeywords={matlab2tikz},
    keywordstyle=\color{blue},%
    morekeywords=[2]{1}, keywordstyle=[2]{\color{black}},
    identifierstyle=\color{black},%
    stringstyle=\color{mylilas},
    commentstyle=\color{mygreen},%
    showstringspaces=false,%without this there will be a symbol in the places where there is a space
    %numbers=left,%
    %numberstyle={\tiny \color{black}},% size of the numbers
    %numbersep=9pt, % this defines how far the numbers are from the text
    emph=[1]{for,end,break},emphstyle=[1]\color{red}, %some words to emphasise
    %emph=[2]{word1,word2}, emphstyle=[2]{style},
}

\setlength\parindent{0pt}

\setcounter{chapter}{5}
\setcounter{secnumdepth}{3}
\setcounter{figure}{12}


\pagestyle{fancy}
\fancyhf{}
\lhead{GPET Versuch 5}
\rhead{Tim Luchterhand, Paul Nykiel}
\cfoot{\thepage}

\author{Tim Luchterhand, Paul Nykiel \protect\\ tim.luchterhand@uni-ulm.de, paul.nykiel@uni-ulm.de}
\title{GPET Versuch 5 --- Hochspannung mit Stanley and Tesla}
\subtitle{Gruppe: Dienstag14}

\begin{document}
        \maketitle
        \section{Spannungsübersetzung beim unbelasteten Transformator}
        \paragraph{Aufgabe}
        In diesem ersten Versuchsteil sollen verschiedene Übersetzungsverhältnisse des
        unbelasteten Transformators betrachtet werden. Bauen Sie dazu die Schaltung mit $R_V = 100\si{\ohm}$
        nach Abbildung~\ref{fig:abb8} auf. Erzeugen Sie ein Sinuseingangssignal $U_{Ein}$ mit Hilfe des Signalgenerators
        des Oszilloskops, $f = 500\si{\hertz}$, $U = 1\si{\volt}_{pp}$.

        \vspace{0.5cm}

        Messen Sie jeweils für die verschiedenen Windungsverhältnisse gemäß Tabelle \ref{tab:mess1} die
        Primärspannung $U_1$, die Sekundärspannung $U_2$ mit den beiden Kanälen des Oszilloskops und
        berechnen Sie das Verhältnis $\frac{U_2}{U_1}$. Übernehmen Sie Tabelle~\ref{tab:mess1} in Ihr Protokoll und ergänzen
        Sie die zu ermittelnden Größen. Diskutieren Sie Ihre Ergebnisse.

        \begin{center}
            \begin{figure}[H]
                \includegraphics[width=\textwidth]{aufgabenBilder/abbildung8.png}
                \caption{Versuchsaufbau des ersten Teilversuchs: unbelasteter Transformator}
                \label{fig:abb8}
            \end{figure}
        \end{center}

        \paragraph{Protokoll}
        $ $
        \begin{table}[h!]
            \begin{center}
                \pgfplotstabletypeset[
                multicolumn names, % allows to have multicolumn names
                col sep=space, % the seperator in our .csv file
                %string replace*={inf}{$\infty$},
                %string type,
                header=false,
                display columns/0/.style={
                column name=$N_1$, % name of first column
                column type={S},string type},  % use siunitx for formatting
                display columns/1/.style={
                column name=$N_2$,
                column type={S},string type},
                display columns/2/.style={
                column name=$U_1$,
                column type={S},string type},
                display columns/3/.style={
                column name=$U_2$,
                column type={S},string type},
                display columns/4/.style={
                column name=$\dfrac{U_2}{U_1}$,
                column type={S},string type},
                every head row/.style={
                before row={\toprule}, % have a rule at top
                after row={
                 & & $\si{\volt}$ & $\si{\volt}$\\
                \midrule} % rule under units
                },
                every last row/.style={after row=\bottomrule},
                ]{mess1.csv}
                \caption{Messtabelle zu Versuch 5.1}
                \label{tab:mess1}
            \end{center}
        \end{table}

        %TODO Diskutieren sie ihr Ergebniss
        % U_2/U_1 sollte gleich N_2 / N_1 sein

        \section{Der belastete Transformator}
        \paragraph{Aufgabe}
        In diesem Versuchteil soll der Einfluss verschiedener Lastwiderstände auf die
        Spannungstransformation bei zwei unterschiedlichen Windungsverhältnissen untersucht werden.
        Bauen Sie dazu die Schaltung mit $R_V = 100\si{\ohm}$ nach Abbildung~\ref{fig:abb9} auf. Als Last $R_L$ verwenden Sie ein
        Potentiometer ($0\ldots1000\si{\ohm}$) dem Sie zusätzlich einen $100\si{\ohm}$ Widerstand in Reihe schalten.
        Als Eingangssignal $U_{Ein}$ stellen Sie am Signalgenerator des Oszilloskops ein Sinussignal
        der Spannung $1 \si{\volt}_{pp}$ und der Frequenz $f = 500\si{\hertz}$ ein.

        \vspace{0.5cm}

        Messen Sie für die Windungsverhältnisse $N_1 : N_2 = 500 : 500$ und $N_1 : N_2 = 250 : 500$
        die Primärspannung $U_1$ sowie die Sekundärspannung $U_2$ für verschiedene Lastwiderstände
        ($100\ldots1000\si{\ohm}$) gemäß Tabelle~\ref{tab:mess2} und berechnen Sie zudem das Verhältnis $\frac{U_2}{U_1}$.
        Übernehmen Sie Tabelle~\ref{tab:mess2} in ihr Protokoll und ergänzen Sie die zu ermittelnden Größen.

        \begin{center}
            \begin{figure}[H]
                \includegraphics[width=\textwidth]{aufgabenBilder/abbildung9.png}
                \caption{Versuchsaufbau des zweiten Teilversuchs: belasteter Transformator}
                \label{fig:abb9}
            \end{figure}
        \end{center}

        Tragen Sie für die beiden Windungsverhältnisse $N_1 : N_2 = 500 : 500$ und $N_1 : N_2 = 250 : 500$
        das Spannungsübersetzungsverhältnis $U_2 / U_1$ über den Lastwiderstand $R_L$ auf und
        diskutieren Sie ihre Ergebnisse.

        \paragraph{Protokoll}
        $ $
        \begin{table}[H]
            \begin{center}
                \pgfplotstabletypeset[
                multicolumn names, % allows to have multicolumn names
                col sep=space, % the seperator in our .csv file
                %string replace*={inf}{$\infty$},
                %string type,
                header=false,
                display columns/0/.style={
                column name=$R_L$, % name of first column
                column type={S},string type},  % use siunitx for formatting
                display columns/1/.style={
                column name=$U_1$,
                column type={S},string type},
                display columns/2/.style={
                column name=$U_2$,
                column type={S},string type},
                display columns/3/.style={
                column name=$U_2 / U_1$,
                column type={S},string type},
                display columns/4/.style={
                column name=$U_1$,
                column type={S},string type},
                display columns/5/.style={
                column name=$U_2$,
                column type={S},string type},
                display columns/6/.style={
                column name=$U_2 / U_1$,
                column type={S},string type},
                every head row/.style={
                before row={\toprule \\ & \multicolumn{3}{c}{$500:500$} & \multicolumn{3}{c}{$250:500$}\\}, % have a rule at top
                after row={
                \si{\ohm} & \si{\volt} & \si{\volt} & & \si{\volt} & \si{\volt}\\
                \midrule} % rule under units
                },
                every last row/.style={after row=\bottomrule},
                ]{mess2.csv}
                \caption{Messtabelle zu Versuch 5.2}
                \label{tab:mess2}
            \end{center}
        \end{table}

        %TODO Ergebnisse Diskutieren

        \section{Kopplungsgrad}
        \paragraph{Aufgabe}
        In diesem Versuchsteil soll der Kopplungsgrad k des Transformators ($N_1 : N_2 = 500 : 500$)
        bei den Frequenzen $f_1 = 500\si{\hertz}$ und $f_2 = 5000\si{\hertz}$ mit Hilfe der in Abschnitt 4.1
        bestimmten Ausdrücke für $L_1$, $L_2$ und $M$ bestimmt werden. (Falls Sie diese Vorbereitungsaufgabe
        nicht lösen konnten, sprechen Sie mit Ihrem Tutor). Zur Berechnung von $L_1$ und $M$
        verwenden Sie den Aufbau nach~\ref{fig:abb10} um die Größen $U_1$ und $I_1$ zu ermitteln. Als
        Eingangssignal erzeugen Sie wieder einen Sinus mit $U = 1\si{\volt}_{pp}$ mit Hilfe des
        Signalgenerators des Oszilloskops. ($R_V = 100\si{\ohm}$)

        \begin{center}
            \begin{figure}[H]
                \includegraphics[width=\textwidth]{aufgabenBilder/abbildung10.png}
                \caption{Aufbau zur Ermittlung des Kopplungsgrades}
                \label{fig:abb10}
            \end{figure}
        \end{center}

        Für die Berechnung von $L_2$ muss der Primärstrom $I_1$, sowie der Kurzschlussstrom $I_2$
        ermittelt werden. Bauen Sie hierzu die Schaltung nach~\ref{fig:abb11} auf. Geben Sie hier
        einen Sinus der Spannung $U = 2\si{\volt}_{pp}$ auf die Schaltung und messen Sie die beiden Ströme
        $I_1$ und $I_2$ mit Hilfe der schwarzen VOLTCRAFT-Multimeter (Messbereich mA, Wech-
        selstrom) für die beiden Frequenzen $f_1$ und $f_2$. Beachten Sie, dass das Multimeter beim
        Messen von Wechselgrößen den Effektivwert liefert!

        \begin{center}
            \begin{figure}[H]
                \includegraphics[width=\textwidth]{aufgabenBilder/abbildung11.png}
                \caption{Versuchsaufbau für die Kurzschlussmessungen zur Bestimmung des Kopplungsgrades}
                \label{fig:abb11}
            \end{figure}
        \end{center}

        Nachdem Sie die Größen $L_1$, $L_2$ und $M$ ermittelt haben, berechnen Sie den Kopplungsgrad
        $k$ für die Frequenzen $f_1$ und $f_2$ und diskutieren Sie Ihre Ergebnisse.

        \paragraph{Protokoll}
        %TODO todo machen

        \section{Frequenzabhängiges Übertragungsverhalten und Phasenschiebung unter Last}
        \paragraph{Aufgabe}
        \begin{center}
            \begin{figure}[H]
                \includegraphics[width=\textwidth]{aufgabenBilder/abbildung12.png}
                \caption{Versuchsaufbau zur Messung des Übertragungsverhalten bei verschiedenen Frequenzen.}
                \label{fig:abb12}
            \end{figure}
        \end{center}

        In diesem letzten Aufgabenteil soll das frequenz- und lastabhängige Übertragungsverhalten
        des Transformators untersucht werden.

        \vspace{0.5cm}

        Messen Sie hierzu mit dem in~\ref{fig:abb12} gezeigten Versuchsaufbau (Transformator mit
        $N_1 = N_2 = 500$ Windungen) die Spannungen $U$ ein und $U_L$ mit Hilfe der beiden Eingangskanäle
        des Oszilloskops als Funktion der Frequenz für die drei Lastwiderstandswerte
        $R_L = \{100\si{\ohm}, 680 \si{\ohm}, 1 \si{k\ohm}\}$. Verwenden Sie hierfür die Sweep-Funktion der MATLAB GUI.
        Setzen Sie die Quellenspannung $U_Q$ mit Hilfe der MATLAB GUI auf $1\si{\volt}$ (Z LOAD = high-Z).
        Nehmen Sie den Betrag und die Phase des Frequenzgangs der Übertragungsfunktion
        $U_2 /U_1 (\omega)$ für Frequenzen zwischen $100\si{\hertz}$ und $100\si{k\hertz}$ (logarithmisch verteilt mit 100
        Frequenzpunkten) für die drei oben genannten Lastwiderstandswerte auf (Hinweis:
        Ermitteln Sie zunächst den Amplitudengang mit der Funktion \glqq{}Sweep --- Frequency\grqq{} und
        anschließend den Phasengang mit der Funktion \glqq{}Sweep --- Phase\grqq{}. Verwenden Sie bei der
        Ermittlung des Phasengangs zur Verbesserung der Qualität der Phasenmessung 4 Mittelungen).

        \vspace{0.5cm}

        Tragen Sie Betrag und Phase von $U_2 /U_1 (\omega)$ für die drei Lastwiderstandswerte in
        einem Bode-Diagramm auf. Diskutieren Sie Ihre Messwerte mit Hilfe Ihrer Ergebnisse aus
        Abschnitt 4.2.

        \paragraph{Protokoll}
\end{document}
